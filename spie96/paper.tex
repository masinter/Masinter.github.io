\documentstyle[spie,psfig]{article}

\title{Multipage Document Images on the Internet}
\author{Les Niles, Gary Kopec, Larry Masinter\\
Xerox Palo Alto Research Center\\
3333 Coyote Hill Road\\
Palo Alto, CA 94304\\
\{niles,kopec,masinter\}@parc.xerox.com}

\date{}
 
\begin{document}
\maketitle

\begin{abstract}
While client/server document imaging systems have matured considerably,
fully satisfactory mechanisms for distributing and
providing interactive access to document images over the World-Wide Web
have not yet emerged. The interface functionality of most scanned document
viewers and browsers is primitive compared to what is available for
revisable-form electronic documents. Common image viewers provide only scrolling within a
page, change of magnification and jumping to the next/previous page.  By
contrast, electronic document browsers often provide content-based
operations such as string search with highlighting of search hits,
up-down-next-previous navigation through logical structure trees, and hypertext
links from indexes and tables of contents to body text. Recently,
there have been a number of efforts aimed at enlivening imaged documents by
providing more content-based interfaces. Examples include Adobe Capture,
Dienst, Xerox's DocuWeb, and the UC Berkeley multivalent
document browser. This paper reviews some of the methods currently used
for transmitting and browsing page images of documents on the Internet and
presents a design for adding some desirable features to 
future document image browsers.
\end{abstract}

\section{Introduction}

While client/server document imaging systems have matured considerably
over the past five years, fully satisfactory mechanisms for
distributing and providing interactive access to document images over
the World-Wide Web have not yet emerged. The interface functionality
of most scanned document viewers and browsers is primitive compared to
what is available for revisable-form electronic documents. Common image viewers
provide only scrolling within a page, change of magnification and
jumping to the next/previous page.  By contrast, electronic document
browsers often provide content-based operations such as string search
with highlighting of search hits, up-down-next-previous navigation
through logical structure trees, and hypertext links from indexes and
tables of contents to body text.

A number of current and planned digital library projects are
emphasizing the use of scanned document material.  Examples include
the ARPA-sponsored Computer Science Technical Report project\cite{cstr} and
the Berkeley environmental digital library\cite{berk}. In the absence of
unexpected and dramatic improvements in document image analysis
technology, a primary means of presenting this material to a user will
be by display of the bitmap images. Thus, improvements to current
image browsing are called for.

\section{Requirements for Usable Image Interfaces}

\subsection{Usability}

While there are numerous activities that might involve image viewing,
the requirements for usability depend significantly on the
application. For example, an application that concentrates on image
viewing of scanned photographic images might contain a number of darkroom-like operations
that modify the view of the photo.  An application that is intended
for preprint preparation might feature operations for creating new combinations
of old material, selection of subsections for reprint, or modification of page numbers.  In 
most digital library applications, reading, searching, and extracting
relevant information from the page images are most important.

\subsection{Performance}

While all applications should be as fast as possible, different
applications have different performance requirements for particular operations. One
important concern for the operation of imaging systems across the
Internet is that it is necessary to compensate for the uneven latency and bandwidth
available across the public network. For this reason, pipelined
operations, where initial pages are retrieved and viewed before
subsequent pages are available, are important. It is also reasonable to
perform image analysis at the server rather than retrieving page
images to the client, e.g., for search.

\subsection{Compatibility}

The image viewer should be compatible with other infrastructure
elements at multiple levels. For example, the use of the Internet
protocols such as HTTP and HTML allows the document provider to offer
access to a wide variety of platforms. The ability to create
postscript allows printing on a wide variety of printers. Other areas
of compatibility include support for multiple document models (page,
chapter, book, pack), and common search operations.

\section{Survey of document imaging on the Internet}

This section describes a sample of the multi-page document image
browsers in use in Internet applications.

The Mercury Image Viewer from CMU\cite{merc} (Fig.\ \ref{mercury}) uses a fast Group 4
decompressor combined with optimized X-window display to give a very rapid display of
document images. Unfortunately, there are no controls for searching.
In the interest of performance, it only offers scaling by integer factors
(1/2, 1/3, 1/4) using simple decimation without anti-aliasing. It uses standard
file system access (AFS, NFS) for accessing individual images. Many
commercial document imaging systems contain a similar mechanism.


The Xerox DocuWeb product gives a web interface to document
repositories stored in a Xerox Documents on Demand (XDOD) library. The
image viewing capability of DocuWeb is accomplished by using embedded
GIF images within HTML coded control structures, as illustrated (Fig.\ \ref{docuweb}). A
view of the entire document is afforded by using individual small
images (``thumbnails'') in a document overview.

The Dienst system\cite{cstr} was developed as part of the Networked
Computer Science Technical Reports Library and contains a
similar interface for displaying page images.

The HyperOCR representation of the Berkeley Digital Library project\cite{berk}
extends this by capturing the OCR of the document and linking each
OCRed page to the original image.  This allows the combination of
searching and image viewing.

Adobe Acrobat contains a viewer for files represented in Adobe's Portable
Document Format (PDF). While PDF is a resolution and
device-independent representation for final-form documents of all
sorts, it can be used for document images through the Adobe Capture
product, which converts images into PDF. The result is a structure
that can be browsed with standard Acrobat viewers (Fig.\ \ref{acrobat}).

\section{Improving Usability using Dil/Bert}

The usability of digital libraries of scanned document image material would
be improved by image viewers and browsers that provide a content-based
interface while preserving the use of scanned images for presentation. This
requires using image analysis algorithms to construct a layer of structured
document representation that is linked to the given presentation image.  A
closely related idea underlies ``Image EMACS,'' an editor for scanned images
of text\cite{imagemacs}.

The structuring and geometric information necessary to support some
useful content-based interface operations is extracted by many OCR
systems for internal use, but is not usually exported. An exception
among commercial products is Xerox Imaging Systems' ScanWorX, which
generates ``XDOC'' files in which the information is encoded. However,
the XDOC representation is oriented toward the specific operation of
the ScanWorX program rather than being a portable external
representation of document structure.

We are developing an interchange representation and document model that
supports content-based interfaces to scanned documents.
In this view, a ``document'' is a unique entity that may have many
different embodiments or take on different forms, for example a scanned
image, pure text from OCR, formatted text from OCR with layout analysis,
logically-structured image blocks from image layout analysis, etc.  Our
top-level document model can contain any or all of these {\it views\/}.
Furthermore, the representation is clearly and completely defined, and
is independent of the source of that information: Formatted text may be
generated by conversion from XDOC, or from the output of some other OCR
software, or by manual markup of plain OCR text, or directly from a word
processor source file; all of these are represented in the same way.

In essence, the representation is intended as the lingua franca between
content-based viewing and manipulation of documents, and the analysis
software that extracts that content in the first place.  This interface
should be clearly and publicly defined, in order to allow a high
degree of interoperability between producers and consumers of document
content information.

Besides serving as an interchange medium, this representation
facilitates caching of analysis results.  If some content-based
interface to a document requests the OCR text of a scanned-image
document, the document ``server'' can arrange to perform OCR when
the text is not already available.  But since OCR is a
computationally-intensive operation, it is desirable to perform it only
once; our model allows the OCR results to simply be added as another
view of the existing document, so if requested again the text is
immediately available.

There are a number of types of analyses in addition to OCR that will be
useful, such as:
\begin{itemize}
\item
Physical layout, obtained either from OCR output such as XDOC, or from
some other software that performs layout analysis but not actual text
recognition (as is done in Image EMACS).

\item
The logical structure of the document: chapters, sections, paragraphs,
and so on.

\item
Keywords, obtained from analysis of the OCRed text.

\item
A summary of the document.

\item
An index, either by OCRing an index in the document and linking the page
references, or by building an index from the logical structure analysis.

\item
Various types of side information, both logical such as the language the
document is written in, and physical such as the specific fonts
the document image is rendered in.

\end{itemize}

The document representation should allow for a wide variety of types of
data, ranging from binary image data to text to complex, virtually
arbitrary, data structures. 
Furthermore, new views of the document and types of analysis data must
be accommodated in the future.

To these ends, we begin with a general-purpose language for defining
data structures and for storing particular instances of those
structures.
In this language a structure is defined for each particular type of data
to be included in the document model.  

The data structures and data are expressed in our Decoder Interchange
Language (DIL).  It provides a set of fundamental types: bit, integer,
real, and character.  It also has multidimensional arrays, one
dimensional sequences, structures, and unions; the elements of these
derived types can be any fundamental or derived type.  DIL has a
syntax for defining the derived types, and for defining instances of
data types and giving them values.

DIL actually has two syntactic forms: ASCII and binary representations.
The ASCII representation is intended to be human-readable, and is
primarily used for designing and interpreting data structures.  This
syntax is somewhat like C, or the Interface Specification Language of
ILU\cite{ilu} (but without any procedural syntax).  DIL does allows
forward references both in declarations and instance definitions.

For actual storage of documents there is no need for human readability,
and the overhead of transferring additional bytes and parsing an ASCII
representation is undesirable.  Therefore DIL also has a binary form,
which is basically just a byte-stream encoding of the data structures
that the ASCII-form parser builds.

The formal document model is specified as a set of DIL declarations.
There is a top-level document structure, which is a sequence of any of
the various views or analysis data that are defined.  The set of
declarations for the top-level document and all the forms that it may
take is named ``BERT,'' and thus the
complete representation is ``Dil/Bert.''

There are a number of obvious document views that we will initially
design Bert specifications for:
binary and gray-scale images; simple OCR text; physical layout
decomposition; logical structure; and formatted OCR text.

To illustrate the sort of content-based document access that Dil/Bert is
designed to enable, we built a prototype document image browser (Fig.\
\ref{ImageBrowser}).  It allows searching for and highlighting
occurrences of a particular word in a document image.  The desired word
can be entered either by typing or by clicking on an instance of that
word.  Multipage documents are fully supported.

This functionality is similar to that provided by
Adobe Capture/Acrobat. However, with Adobe Capture, the result is a
single encoding of the image and content-analysis intermixed, which
supports a limited range of content-based access applications.
By producing a separate representation and dealing with it
orthogonally, Dil/Bert allows the same source material to be used for
other kinds of analysis, e.g., retrieval by form, segmentation, or
image similarity.
Because Dil/Bert is designed to be
extended in a structured manner to include new types of content,
it allows virtually everything that can be derived from a document to be
represented parsimoniously.

In many respects our multiple-view model of a document is similar to the
``multivalent document'' described by Phelps and Wilensky\cite{berk2} of the
UC Berkeley Digital Library Project.  Our work is focused more on the
storage and representation of the document, rather than on the interface
to it; indeed, Dil/Bert may often be used for storing document-related
information which no human user would be interested in interacting with.
We also believe it will be organizationally advantageous to store all
components of a document in a single file, but in a way that makes it
easy for a server to extract and deliver only those components needed
for a particular purpose; we're in agreement with the Berkeley group
that the representation should not force one to always deal with the
document as an inseparable, unified mass.  We also agree that the
representation should fundamentally be extensible, both in allowing for
new types of views to be defined and in allowing for the views of a
particular document to be added in stages.

\section{Acknowledgments}

The authors thank Damon Liu, who was involved in the initial Dil/Bert
design and who wrote the document browser shown in Fig.\
\ref{ImageBrowser}.

\begin{thebibliography}{99}

\bibitem{imagemacs}Bagley, S.C. and G.E. Kopec, ``Editing images of text,''
{\it CACM\/}, pp. 63-72, December 1994. 

\bibitem{ilu}Courtney, A., W. Janssen et al., ``Inter-Language Unification, release 1.5,'' Xerox PARC Technical Report P94-00058, May 1994. See also {\tt ftp://parcftp.parc.xerox.com/pub/ilu/ilu.html}.

\bibitem{cstr}Davis, J.R., ``Creating a Networked Computer Science
Technical Report Library,'' {\it D-Lib Magazine\/}, September 1995.  See
also {\tt http://WWW.CNRI.Reston.VA.US/home/cstr.html}.

\bibitem{berk2}Phelps, T.A. and R. Wilensky, Multivalent Documents: Inducing
Stucture and Behaviors in Online Digital Documents, {\it Proc. 29th
Annual Hawaii International Conf. on System Sciences\/}, pp. 144-152,
1996. 

\bibitem{panel}Thoma, G., et al., Panel Report, ``Access to Document Images over the Internet,'' Computers in Healthcare Education Symposium, Thomas Jefferson University, April, 1995. Available at {\tt http://aisr.lib.tju.edu/ CWIS/OAC/hslc/sym95/thoma.html}.

\bibitem{berk}Wilensky, R., UC Berkeley's Digital Library Project,
{\it CACM\/}, 38(4):60, April 1995.

\bibitem{merc}``Project Mercury and Development of the Library Information System,'' Mercury Technical Report Series, Carnegie Mellon University, 1993.

\end{thebibliography}
\vskip2in

\begin{figure}
\center{\vrule height0pt\hbox{\fbox{\psfig{figure=psfigs/merc5.ps}}}}%%
\caption
{The CMU image browser is fast but has only simple controls.}
\label{mercury}
\end{figure}

\begin{figure}
\center{\vrule height0pt\hbox{\fbox{\psfig{figure=psfigs/acrobat5.ps}}}}%%
\caption
{The Adobe Acrobat reader offers structured navigational aids.
Document images converted with Adobe Capture can be searched for text,
as well as displayed in a format that is faithful to the original
image.}
\label{acrobat}
\end{figure}

\begin{figure}
%\center{\vrule height0pt\hbox{\fbox{\psfig{figure=psfigs/xdodpage5.ps}}}}%%
%\center{\vrule height0pt\hbox{\fbox{\psfig{figure=psfigs/xdodmenu5.ps}}}}%%
\center{\vrule height0pt\hbox{\fbox{\vbox
{\psfig{figure=psfigs/xdodpage5.ps}\psfig{figure=psfigs/xdodmenu5.ps}}}}}%%
\caption
{Xerox Document on Demand page image and thumbnail menu.}
\label{docuweb}
\end{figure}

\begin{figure}
\center{\vrule width0pt\hbox{\fbox{\psfig{figure=psfigs/imgbrwsr5.ps}}}}%%
\caption
{A document image browser, illustrating one application of Dil/Bert.}
\label{ImageBrowser}
\end{figure}

\end{document}
